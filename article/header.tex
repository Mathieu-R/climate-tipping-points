%\usepackage[colorlinks]{hyperref} % colored hyperlink
\usepackage{amsmath}
\usepackage{amssymb}
\usepackage{amsthm}
\usepackage[T1]{fontenc}
\usepackage[french]{babel} % traduction
\usepackage[usenames,svgnames,dvipsnames]{xcolor}
%\usepackage{bookmark} 
\usepackage{booktabs} % beautiful tables
\usepackage[thicklines]{cancel} % cancel terms in equation
\usepackage{color}
\usepackage{enumerate}
\usepackage{framed} % box around text
\usepackage{fancyhdr} % header
\usepackage{fancyvrb}
\usepackage{mathtools}
\usepackage[parfill]{parskip}
\usepackage{pgfplots} % plot in tikz
\usepackage[arrowdel]{physics} % physics helpers
\usepackage{setspace}
\usepackage{siunitx}
\usepackage{silence} % silence warnings
\usepackage{systeme} % system of equations
\usepackage{tikz}
\usepackage{tkz-euclide}
\usepackage{tikz-3dplot}
\usepackage{units} % Non-stacked fractions and better unit spacing
\usepackage{xspace} % prints a trailing space in a smart way.

\usetikzlibrary{calc,patterns,angles,quotes}

\tikzset{circ/.style = {fill, circle, inner sep = 0, minimum size = 3}}
\tikzset{scirc/.style = {fill, circle, inner sep = 0, minimum size = 1.5}}
\tikzset{mstate/.style={circle, draw, blue, text=black, minimum width=0.7cm}}

\usepackage{enumitem}
\setlist{nosep,after=\vspace{\baselineskip}}

\pgfplotsset{compat=1.12}
\sisetup{locale = FR}

\usepackage[marginparwidth=2cm]{geometry}
\geometry{
	paper=a4paper,
	inner=2.0cm,
	outer=3.0cm,
	bindingoffset=.5cm, 
	top=3.5cm, 
	bottom=3.5cm, 
}

\usepackage{graphicx}
\setkeys{Gin}{width=\linewidth,totalheight=\textheight,keepaspectratio}
%\graphicspath{{figures/}}

% Default images settings
\setkeys{Gin}{width=\linewidth, totalheight=\textheight, keepaspectratio}

\definecolor{blue}{rgb}{0,0,1}
\definecolor{red}{rgb}{1,0,0}

%------------------
% Théorèmes,...
%------------------
\usepackage{thmtools}
\usepackage[framemethod=TikZ]{mdframed}

\mdfdefinestyle{mdgreenbox}{%
	skipabove=8pt,
	linewidth=2pt,
	rightline=false,
	leftline=true,
	topline=false,
	bottomline=false,
	linecolor=ForestGreen,
	backgroundcolor=ForestGreen!5,
}
\declaretheoremstyle[
	headfont=\bfseries\sffamily\color{ForestGreen!70!black},
	bodyfont=\normalfont,
	postheadspace=\newline,
	spaceabove=2pt,
	spacebelow=1pt,
	mdframed={style=mdgreenbox},
	headpunct={ --- }
]{thmgreenbox}

\mdfdefinestyle{mdblackbox}{%
	skipabove=8pt,
	linewidth=3pt,
	rightline=false,
	leftline=true,
	topline=false,
	bottomline=false,
	linecolor=black,
	backgroundcolor=RedViolet!5!gray!5,
}
\declaretheoremstyle[
	headfont=\bfseries,
	bodyfont=\normalfont\small,
	spaceabove=0pt,
	spacebelow=0pt,
	mdframed={style=mdblackbox},
	postheadspace=\newline
]{thmblackbox}

\mdfdefinestyle{mdbluebox}{%
	roundcorner = 10pt,
	linewidth=1pt,
	skipabove=12pt,
	innerbottommargin=9pt,
	skipbelow=2pt,
	nobreak=true,
	linecolor=blue,
	backgroundcolor=TealBlue!5,
}
\declaretheoremstyle[
	headfont=\sffamily\bfseries\color{MidnightBlue},
	mdframed={style=mdbluebox},
	headpunct={\\[3pt]},
	postheadspace=\newline
]{thmbluebox}

\mdfdefinestyle{mdredbox}{%
	linewidth=0.5pt,
	skipabove=12pt,
	frametitleaboveskip=5pt,
	frametitlebelowskip=0pt,
	skipbelow=2pt,
	frametitlefont=\bfseries,
	innertopmargin=4pt,
	innerbottommargin=8pt,
	nobreak=true,
	linecolor=RawSienna,
	backgroundcolor=Salmon!5,
}
\declaretheoremstyle[
	headfont=\bfseries\color{RawSienna},
	mdframed={style=mdredbox},
	headpunct={\\[3pt]},
	postheadspace=\newline,
]{thmredbox}

\theoremstyle{definition}
\declaretheorem[name=Théorème,numberwithin=section,style=thmredbox]{theorem}
\declaretheorem[name=Définition,sibling=theorem,style=thmgreenbox]{definition}
\declaretheorem[name=Proposition,sibling=theorem,style=thmbluebox]{proposition}
\declaretheorem[name=Corollaire,sibling=theorem,style=thmbluebox]{corollary}
\declaretheorem[name=Lemme,sibling=theorem,style=thmbluebox]{lemma}
\declaretheorem[name=Exemple,sibling=theorem,style=thmblackbox]{example}
\declaretheorem[name=Question,sibling=theorem,style=thmblackbox]{ques}
\declaretheorem[name=Exercice,sibling=theorem,style=thmblackbox]{exercise}
\declaretheorem[name=Remarque,sibling=theorem,style=thmgreenbox]{remark}
\declaretheorem[name=Etape,style=thmgreenbox]{step}

% \theoremstyle{plain}
% \newtheorem{proof}{Preuve}

% conditions for equations
% https://tex.stackexchange.com/questions/95838/how-to-write-a-perfect-equation-parameters-description
\newenvironment{conditions}
  {\par\vspace{\abovedisplayskip}\noindent\begin{tabular}{>{$}l<{$} @{${}={}$} l}}
	{\end{tabular}\par\vspace{\belowdisplayskip}}
	
\newenvironment{conditions*}
  {\par\vspace{\abovedisplayskip}\noindent
   \tabularx{\columnwidth}{>{$}l<{$} @{${}={}$} >{\raggedright\arraybackslash}X}}
  {\endtabularx\par\vspace{\belowdisplayskip}}

%----------------
% FIX
%----------------

\setlength{\headheight}{14.5pt}

% increase vertical space for aligned equations
\setlength{\jot}{7pt}

% Filter warnings issued by package biblatex starting with "Patching footnotes failed"
\WarningFilter{biblatex}{Patching footnotes failed}

%----------------------
%	COMMANDS
%----------------------

% commands shortcuts
\newcommand{\mb}{\mathbb}
\newcommand{\R}{\mb{R}}
\newcommand{\Z}{\mb{Z}}
\newcommand{\N}{\mb{N}}
\newcommand{\C}{\mb{C}}
\newcommand{\dS}{\cdot d\vec{S}}
\newcommand{\lag}{\mathcal{L}}
\newcommand{\vd}[1]{\dot{\va{#1}}}
\newcommand{\vdd}[1]{\ddot{\va{#1}}}

% norm
\newcommand{\bignorm}[1]{\left\lVert#1\right\rVert}

% prints an asterisk that takes up no horizontal space.
% useful in tabular environments.
\newcommand{\hangstar}{\makebox[0pt][l]{*}}

% Prints argument within hanging parentheses (i.e., parentheses that take
% up no horizontal space). Useful in tabular environments.
\newcommand{\hangp}[1]{\makebox[0pt][r]{(}#1\makebox[0pt][l]{)}}

% cancel terms with color
\newcommand{\ccancel}[2]{\renewcommand{\CancelColor}{\color{#2}}\bcancel{#1}}

\renewcommand*\contentsname{Table des matières}