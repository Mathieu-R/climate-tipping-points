\section{Bifurcation de fold-hopf}

Cette bifurcation apparait dans un système d'équations différentielles ordinaires (EDOs) constitué à la fois d'une EDO à 1 dimension que l'on appellera système primaire et d'un sous-système d'EDOs à 2 dimensions que l'on appellera système secondaire. De ce premier système surgit une bifurcation de type \emph{"Fold"} tandis que du second système surgit une bifurcation de type \emph{"Hopf"}.

\begin{equation}
  \begin{cases}
    \dot{x} = a_1x^3 + a_2x + \phi \quad, \textit{système primaire} \\
    \dot{y} = b_1z + b_2(\gamma(x) - (y^2 + z^2))y \quad, \textit{système secondaire} \\
    \dot{z} = c_1y + c_2(\gamma(x) - (y^2 + z^2))z
  \end{cases}
\end{equation}

Alors que nous nous approchons de la valeur critique de la bifurcation que nous noterons $\phi_{crit}$, une petite perturbation du paramètre $phi$ suffit pour passer la bifurcation et modifier considérablement le comportement qualitatif du système. Dans le langage des systèmes dynamiques on dit qu'il y a création, suppression d'équilibres voir modification dans la nature de ces équilibres. Toutefois, dans la définition des points de bascules, il en existe certains qui sont reversibles et qui ne sont donc pas dû à des bifurcations (\cite{lenton_tipping_2008}). Nous ne considérerons pas ces derniers dans cet article.

\subsection{Bifurcation fold}

Une bifurcation fold est une combinaison de 2 bifurcations points de selles

