\section{Conclusion}
Après avoir introduit le concept de points de bascule et leur importance cruciale dans l'étude du système climatique ainsi qu'en considérant ce dernier comme un grand système dynamique nous avons vu que nous pouvions modéliser des sous-systèmes climatiques sous forme d'équations différentielles ordinaires. Nous nous sommes concentrés en particulier sur un système comprenant une bifurcation de fold-hopf. Nous avons vu que l'on pouvait décomposer ce système en un système primaire et secondaire et que ces deux derniers étaient alors couplés par un paramètre de couplage $\gamma(x)$. En faisant lentement varier le paramètre de forçage $\phi$ dans le système primaire, on a alors vu que la bascule de ce dernier provoquait la bascule du second par cascade et ainsi la coexistence d'états stables et oscillatoires. On a finalement vu que la présence de bruit pouvait également pousser le système à basculer. Nous avons remarqué que cette bifurcation pourrait éventuellement apparaître dans un modèle décrivant le lien entre l'AMOC et le phénomène El-Niño. Une modélisation simple de ce phénomène est disponible dans l'article de \cite{dekker_cascading_2018}. Toutefois, il s'agit d'un modèle fortement idéalisé et de plus amples recherches sur des modèles plus détaillés sont nécessaires afin de démontrer si de telles bascules sont possibles dans le système climatique.

Comme nous l'avons mentionné en introduction, plusieurs recherches ont mis en évidence la possible existence de points de bascules. Au vu de l'importance et des conséquences de ces derniers sur notre climat et à fortiori sur notre société, ce sujet est d'une importance cruciale et de plus amples recherches sont dès lors nécessaires.

