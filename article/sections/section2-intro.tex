\section{Introduction}

Notre planète subit un changement climatique dont on peut penser que les conséquences apparaîtront graduellement. Pourtant plusieurs recherches ont mis en évidence la possible existence d'\emph{éléments de bascule} dans le système climatique (\cite{lenton_tipping_2008}). Ces éléments contiendraient ce qu'on appelle communément des points de bascule. Si l'on se représente le système climatique terrestre comme un \emph{système dynamique} (dissipatif) alors on peut voir ce point de bascule comme une bifurcation. Il s'agit d'un changement structurel dans le système lorsqu'un paramètre $\phi$ --- dit de forçage --- de ce système atteint une valeur critique $\phi_c$. Alors qu'on approche ces points de bascule à un temps $t$, un petit changement peut suffir à passer ces points et modifier de façon qualitative l'état du système considéré. Cela peut notamment induire des boucles de rétroactions positives ou encore des bifurcations. Par ailleurs, ces transitions peuvent être brusques, irréversibles ou même parfois les deux (\cite{Lenton_2012}). A titre d'exemple d'éléments de bascule nous pourrions citer la banquise Arctique, la banquise du Groenland, la circulation thermohaline Atlantique, la mousson d'été Indienne ou encore la forêt tropicale amazonienne (\cite{lenton_tipping_2008, thompson_predicting_2011}). Ces éléments ne sont pas pas isolés mais intéragissent entre eux à grande échelle (\cite{lenton_climate_2019_too_risky}) et ces interactions pourraient alors augmenter ou diminuer la probabilité d'apparition de bifurcations en cascade (effets domino).

Dans le bassin amazonien, une grande partie des précipitations s'évapore causant de nouvelles précipitations. Etant donné le role important de la végétation dans le cycle hydrologique, des simulations de la déforestation de la forêt tropicale amazonienne indiquent une réduction entre $20$ et $30\%$ de ces précipitations (\cite{zeng_climatic_1996}) ainsi qu'un alongissement de la saison sèche et une augmentation des température estivales. Tout ces éléments mettraient en péril la regénération de cette forêt et suggégeraient que le système dispose d'une bistabilité. En effet, passer un point critique de ce sous-système pourrait convertir de larges zones de la forêt amazonienne en savane ou en régions sèches de façon saisonnière (\cite{steffen_trajectories_2018}). L'approche de ce point critique pourrait à la fois être causé par l'activité humaine, un changement dans la fréquence des feux de forêt, une réduction des précipitations régionales (\cite{lenton_tipping_2008}) mais également un renforcement du phénomêne El-Niño\footnote{Phénomène climatique caractérisé par des températures anormalement élevées de l'eau dans la partie Est de l'océan Pacifique-Sud.} et un ralentissement de la circulation méridionale de retournement Atlantique (AMOC)\footnote{Désigne l'ensemble des courants océaniques (Gulf stream, circulation thermohaline,...) régissant les échanges de chaleur entre l'équateur et les pôles.} conduisant tous les deux à un assèchement plus important du bassin amazonien (\cite{lenton_tipping_2008, lenton_climate_2019_too_risky}). Nous verrons plus loin que ces 2 derniers phénomènes pourraient également être reliés entre eux.

% Si on regarde de plus près la banquise Arctique, alors que la zone couverte par la glace diminue, une surface plus sombre de l'océan est exposée. Cette surface absorbe plus les rayonnements solaires que celle de la neige et la glace ce qui a pour conséquence d'augmenter le réchauffement et donc de diminuer encore plus la surface de glace de mer. Cette couverture de glace possède une boucle de rétroaction positive et pourrait exhiber une bistabilité. Cette dernière pourrait d'ailleurs contenir plusieurs états de couche de glace. Certains chercheurs pensent qu'un seuil de fonte de la banquise d'été serait dépassé ou serait en passe d'être dépassé et qu'une transition vers un nouvel état pourrait avoir lieu ce siècle. D'autres chercheurs pensent plutôt qu'il n'y aurait non pas une bifurcation pour la perte de banquise d'été mais pour la perte de la banquise tout au long de l'année.

% Des recherches récentes montrent que des signes indiqueraient que ces points de bascules soient plus probables qu'il était pensé autrefois et qu'ils impliqueraient des changements irréversibles (\cite{lenton_climate_2019_too_risky}). Dès lors, ces points de bascule constituent une urgence climatique d'autant plus que des récents rapports de l'IPCC parus en 2018 et 2019 suggèrent qu'un réchauffement de 1 à \SI{2}{\celsius} suffirait à dépasser ces points (\cite{ipcc_global_2018,portner_ipcc_2019}). De plus, on serait proche de certains points de bascules et certains sous-systèmes climatiques auraient déjà passé un point de cascade à l'image par exemple de l'échancrure de la mer d'Amundsen en Antarctique Ouest (\cite{portner_ipcc_2019}) où la ligne de rencontre de la glace, la roche et l'océan est en train de reculer de façon irréversible ce qui pourrait déstabiliser le reste de la calotte glaciaire (de l'Antarctique Ouest) par effet domino (\cite{feldmann_collapse_2015_amundsen}). La conséquence est sans appel, une montée du niveau d'eau d'environ 3m s'étalant sur plusieurs centaines voire milliers d'années. Ces points de bascules font peut-être partie des plus grandes menaces climatiques.

L'objectif de cet article est de s'intéresser aux points de bascules en tant que bifurcations et plus particulièrement d'étudier la bifurcation \emph{fold-hopf} et quelles seraient les conséquences pour un système climatique passant cette bifurcation. Premièrement, on définit la théorie des systèmes dynamiques derrière cette bifurcation. Ensuite on analyse les bifurcations de fold et hopf à elles seules et leurs conséquences pour ensuite les coupler et former la bifurcation fold-hopf. Par après, on regarde comment cette bifurcation se passe dans le temps. Enfin, on ajoute du bruit à notre système dynamique et regardons ses conséquences.
