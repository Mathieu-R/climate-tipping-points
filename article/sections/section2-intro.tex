\section{Introduction}

Notre planète subit un changement climatique, il fut un temps où politiques, économistes et même scientifiques ont pensé que le changement climatique induit par l'homme était prédictible et que ses conséquences apparaitraient graduellement, linéairement dans le temps. Pourtant plusieurs recherches ont mis en évidence la possible existence de \emph{points de bascules} dans le système climatique (\cite{lenton_tipping_2008}). Il s'agit d'un changement apparaissant dans un sous-système climatique (dû à la monter des températures par exemple) induisant un changement dans un autre sous-systèmes climatiques. Si l'on se représente le système climatique terrestre comme un \emph{système dynamique} alors on peut voir ce point de bascule comme une bifurcation. Il s'agit d'un changement structurelle dans le système lorsqu'un paramètre $\phi$ de ce système atteint une valeur critique $\phi_c$. Alors qu'on approche ces points de bascule à un temps $t$, un petit changement peut suffir à passer ces points critiques et avoir de graves conséquences. Cela peut notamment induire des boucles de rétroactions positives (qui empirent un changement climatique par exemple) ou encore des bifurcations où le comportement future du système devient instable. D'ailleurs ces transitions peuvent être brusque, irréversibles ou même parfois les deux. Par exemple TODO: exemples de l'Arctique (\cite{Lenton_2012}). A titre d'exemple de points de cascade nous pourrions citer la banquise Arctique, la circulation thermohaline Atlantique ou encore la Forêt Boréal (\cite{lenton_tipping_2008}). Des recherches récentes (TODO: sources) montrent que des signes indiqueraient que ces points de bascules soient plus probables qu'il était pensé autrefois et qu'ils impliqueraient des changements irréversibles (\cite{lenton_climate_2019_too_risky}). Dès lors, ces points de bascules constituent une urgence climatique d'autant plus que des récents rapports de l'IPCC parus en 2018 et 2019 suggèrent qu'un réchauffement de 1 à \SI{2}{\celsius} suffirait à dépasser ces points (\cite{ipcc_global_2018,portner_ipcc_2019}). Pire encore, on serait proche de certains points de bascules et certains sous-sytèmes climatiques auraient déjà passé un point de cascade à l'image par exemple de l'échancrure de la mer d'Amundsen en Antarctique Ouest (\cite{portner_ipcc_2019}) où la ligne de rencontre de la glace, la roche et l'océan est en train de reculer de façon irréversible ce qui pourrait déstabiliser le reste de la calotte glaciaire (de l'Antarctique Ouest) par effet domino (\cite{feldmann_collapse_2015_amundsen}). La conséquence est sans appel, un montée du niveau d'eau d'environ 3m s'étalant sur plusieurs centaines voire milliers d'années. Ces points de bascules font peut-être partie des plus grandes menaces climatiques.

L'objectif de cet article est de s'intéresser aux points de bascules en tant que bifurcations et plus particulièrement d'étudier la bifurcation \emph{fold-hopf} et quelles seraient les conséquences pour un système climatique passant cette bifurcation. Premièrement on définit la théorie des systèmes dynamiques derrière cette bifurcation. Ensuite on analyse les bifurcations de fold et hopf à elles seules et leurs conséquences pour ensuite les coupler et former la bifurcation fold-hopf. Par après on regarde comment cette bifurcation se passe dans le temps. Enfin, on ajoute du bruit à notre système dynamique et regardons ses conséquences.
