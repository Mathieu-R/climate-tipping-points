\section{Introduction}

Notre planète subit un changement climatique dont on peut penser que les conséquences apparaîtront graduellement. Pourtant plusieurs recherches ont mis en évidence la possible existence d'\emph{éléments de bascule} dans le système climatique (\cite{lenton_tipping_2008}). Ces éléments contiendraient ce qu'on appelle communément des points de bascule. Si l'on se représente le système climatique terrestre comme un \emph{système dynamique} alors on peut voir ce point de bascule comme une bifurcation. Il s'agit d'un changement structurel dans le système lorsqu'un paramètre $\phi$ --- dit de forçage --- de ce système atteint une valeur critique $\phi_c$. Alors qu'on approche ces points de bascule à un temps $t$, un petit changement peut suffir à passer ces points et modifier leur état de façon qualitative. Cela peut notamment induire des boucles de rétroactions positives ou encore des bifurcations. Par ailleurs ces transitions peuvent être brusques, irréversibles ou même parfois les deux (\cite{Lenton_2012}). A titre d'exemple d'éléments de bascule nous pourrions citer la banquise Arctique, la circulation thermohaline Atlantique ou encore la forêt boréale (\cite{lenton_tipping_2008}). Des recherches récentes montrent que des signes indiqueraient que ces points de bascules soient plus probables qu'il était pensé autrefois et qu'ils impliqueraient des changements irréversibles (\cite{lenton_climate_2019_too_risky}). Dès lors, ces points de bascule constituent une urgence climatique d'autant plus que des récents rapports de l'IPCC parus en 2018 et 2019 suggèrent qu'un réchauffement de 1 à \SI{2}{\celsius} suffirait à dépasser ces points (\cite{ipcc_global_2018,portner_ipcc_2019}). De plus, on serait proche de certains points de bascules et certains sous-systèmes climatiques auraient déjà passé un point de cascade à l'image par exemple de l'échancrure de la mer d'Amundsen en Antarctique Ouest (\cite{portner_ipcc_2019}) où la ligne de rencontre de la glace, la roche et l'océan est en train de reculer de façon irréversible ce qui pourrait déstabiliser le reste de la calotte glaciaire (de l'Antarctique Ouest) par effet domino (\cite{feldmann_collapse_2015_amundsen}). La conséquence est sans appel, une montée du niveau d'eau d'environ 3m s'étalant sur plusieurs centaines voire milliers d'années. Ces points de bascules font peut-être partie des plus grandes menaces climatiques.

L'objectif de cet article est de s'intéresser aux points de bascules en tant que bifurcations et plus particulièrement d'étudier la bifurcation \emph{fold-hopf} et quelles seraient les conséquences pour un système climatique passant cette bifurcation. Premièrement on définit la théorie des systèmes dynamiques derrière cette bifurcation. Ensuite on analyse les bifurcations de fold et hopf à elles seules et leurs conséquences pour ensuite les coupler et former la bifurcation fold-hopf. Par après on regarde comment cette bifurcation se passe dans le temps. Enfin, on ajoute du bruit à notre système dynamique et regardons ses conséquences.
