\lettrineabstract{
  Nous discutons des points de bascules dans le système climatique. En considérant ce dernier comme un système dynamique, ces points de bascules sont les équivalents des bifurcations de la théorie des systèmes dynamiques. Nous nous penchons particulièrement sur la bifurcation de fold-hopf. Pour cela, nous analysons un système d'équations différentielles où apparait ce type de bifurcation. Premièrement, nous séparons ce système afin d'analyser les bifurcations de fold et hopf séparement pour ensuite analyser la bifurcation fold-hopf. Nous analysons ensuite la série temporelle du système complet d'équations différentielles afin de comprendre comment apparait le phénomène de cascade de points de bascule. Nous ajoutons finalement un bruit gaussien afin de comprendre comment se comporte le système climatique en présence d'une perturbation.
}

  % Les points de bascules qu'ils soient associés à des bifurcations ou provoqués par du bruit peuvent conduire à des changements climatiques brutaux et amener le système climatique à se comporter de manière imprévisible. Ces bascules conduisant à un changement rapide et radical, il deviendrait alors difficile pour la société et pour notre écosystème de s'adapter directement. La recherche de possibles éléments de bascule et leur analyse se révèlent d'une importance capitale. Dans cet article nous analysons en particulier une bifurcation de type fold-hopf et étudions ses conséquences pour le système climatique.
