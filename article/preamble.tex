%-----------------
% HEADER / FOOTER
%-----------------

% header
\chead{\textit{\ntitle}}

% footer
\rfoot{\footnotesize Page \npage\ \\ pageref{LastPage}}

% page style for the first page
\fancypagestyle{firstpage}{
  \fancyhf{}
  \renewcommand{\footrulewidth}{0.0pt} % no footer rule
}

%----------
% TITLE
%----------

% custom title
\usepackage{titling} % Allows custom title configuration

% Gold horizontal line around the title
\newcommand{\HorRule}{\color{DarkGoldenrod}\rule{\linewidth}{1pt}}

% custom title
\pretitle{
	\vspace{-30pt} % Move the entire title section up
	\HorRule\vspace{10pt} % Horizontal rule before the title
	\fontsize{32}{36}\usefont{OT1}{phv}{b}{n}\selectfont % Helvetica
	\color{DarkRed} % Text colour for the title and author(s)
}

\posttitle{\par\vskip 15pt} % Whitespace under the title

\preauthor{} % Anything that will appear before \author is printed

\postauthor{ % Anything that will appear after \author is printed
	\vspace{10pt} % Space before the rule
	\par\HorRule % Horizontal rule after the title
	\vspace{20pt} % Space after the title section
}

%--------------------------
% AUTHORS AND INSTITUTIONS
%--------------------------

% author
\newcommand{\authorstyle}[1]{{\large\usefont{OT1}{phv}{b}{n}\color{DarkRed}#1}} % Authors style (Helvetica)

% institution
%\newcommand{\institution}[1]{{\footnotesize\usefont{OT1}{phv}{m}{sl}\color{Black}#1}} % Institutions style (Helvetica)

%----------
% ABSTRACT
%----------

\usepackage{lettrine} % Package to accentuate the first letter of the text (lettrine)
\usepackage{fix-cm}	% Fixes the height of the lettrine

\newcommand{\initial}[1]{ % Defines the command and style for the lettrine
	\lettrine[lines=3,findent=4pt,nindent=0pt]{% Lettrine takes up 3 lines, the text to the right of it is indented 4pt and further indenting of lines 2+ is stopped
		\color{DarkGoldenrod}% Lettrine colour
		{#1}% The letter
	}{}%
}

\usepackage{xstring} % Required for string manipulation

\newcommand{\lettrineabstract}[1]{
	\StrLeft{#1}{1}[\firstletter] % Capture the first letter of the abstract for the lettrine
	\initial{\firstletter}\textbf{\StrGobbleLeft{#1}{1}} % Print the abstract with the first letter as a lettrine and the rest in bold
}

%--------------
% BIBLIOGRAPHY
%--------------
\usepackage[backend=biber,style=apa,autocite=inline]{biblatex}
\DeclareLanguageMapping{french}{french-apa}
\usepackage[autostyle=true]{csquotes} % Required to generate language-dependent quotes in the bibliography
